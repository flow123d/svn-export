% Copyright (C) 2007 Technical University of Liberec.  All rights reserved.
%
% Please make a following refer to Flow123d on your project site if you use the program for any purpose,
% especially for academic research:
% Flow123d, Research Centre: Advanced Remedial Technologies, Technical University of Liberec, Czech Republic
%
% This program is free software; you can redistribute it and/or modify it under the terms
% of the GNU General Public License version 3 as published by the Free Software Foundation.
%
% This program is distributed in the hope that it will be useful, but WITHOUT ANY WARRANTY;
% without even the implied warranty of MERCHANTABILITY or FITNESS FOR A PARTICULAR PURPOSE.
% See the GNU General Public License for more details.
%
% You should have received a copy of the GNU General Public License along with this program; if not,
% write to the Free Software Foundation, Inc., 59 Temple Place - Suite 330, Boston, MA 021110-1307, USA.

In this section we briefly descripte every option you can use in global INI-file. Options marked
(NOT IMPLEMENTED) are not implemented in current version, but probably will be reimplemented in near future.
Options marked (NOT SUPPORTED) has not been tested since they are obsolete and will be soon replaced by similar functionality.

\begin{initable}{Global}
 \key{Problem\_type} & \type{int} & NULL &
 Type of solved problem. Currently supported:\break %\br 
 1 = steady saturated flow \br
 2 = unsteady saturated flow using MH method \br
 4 = unstedy saturated flow using Lumped MH method 
% 3 = variable-density saturated flow
 \\
 \hline
 \key{Description} & \type{string} & {\it undefined} &
 Short description of solved problem - any text.
 \\
 \hline
 \key{Stop\_time} & \type{double} & 1.0 &
 Time interval of the whole problem.%\br
 [time units]
 \\
 \hline
 \key{Time\_step} & \type{double} & 1.0 &
 Time step for unstedy water flow solver.
 [time units]
 \\
 \hline
 \key{Save\_step} & \type{double} & 1.0 &
 The output with transport is written every
 {\tt Save\_step}. [time units]
 \\
% \hline
% \key{Density\_step} & \type{double} & 1.0 &
% Time interval of one density iteration
% in the varible-density calculation (type=3)
% [time units]
 \\
 \hline
\end{initable}

 
%\begin{initable}{Parallel}
% \key{part\_type} & \type{int} & 0 &
%Partitioning based on (EXPERIMENTAL):
% 1 - GMSH element partitioning\br
% 2 - ParMETIS edges graph of highest dimension\br
% 3 - ParMETIS full edges graph
%\\
%\hline
%\end{initable}

\begin{initable}{Input}
% \key{File\_type} & \type{int} & -1 &
% Type of the input files. Now only the value 1 
% (GMSH-like files) is accepted.
%\\
%\hline
\key{Mesh} & \type{string} & NULL & 
Name of file containig definition of the mesh
for the problem. (see Section \ref{mesh_file})
\\
\hline
\key{Material} & \type{string} & NULL &
Name of file with hydraulical properties of
the elements. (see Section \ref{material_file})
\\
\hline
\key{Boundary} & \type{string} & NULL &
Name of file with boundary condition data. (see Section \ref{boundary_file})
\\
\hline
\key{Neighbouring} & \type{string} & NULL &
Name of file describing topology of the mesh. (see Section \ref{ngh_file})
\\
\hline
\key{Initial} & \type{string} & NULL &
Name of file with initial condition for preasure head [length]. (see Section \ref{element_data_file})
\\
\hline
\key{Sources} & \type{string} & NULL &
Name of file with definition of fluid sources. 
This is optional file, if this key is not
defined, calculation runs without sources. (see Section \ref{element_data_file})
\\
\hline
\key{sources\_formula} & \type{string} & NULL &
Expression for sources as function of space coordinates $x$, $y$, $z$.
See documentation of FParser library:
\url{http://warp.povusers.org/FunctionParser/fparser.html\#literals}
\end{initable}

\begin{initable}{Run}
%\key{Log\_file} & \type{string} & mixhyb.log &
%Name of log file.
%\\ 
%\hline
\key{Screen\_verbosity} & \type{int} & 0 &
Nonzero value turn on a more verbose mode (more messages on screen), however everything is in the log file.
\\
\hline
\key{Pause\_after\_run} & \type{YES/NO} & NO &
Wait for Enter after end of progam in order to keep output screen open.
\\
\hline
\end{initable}

%%%%%%%%%%%%%%%%%%%%%%%%%%%%%%%%%%%%%%%%%%%%%%%%%%%%%%%%%%%%%%%%%%%%%%%%5


\begin{initable}{Transport}
 \key{Transport\_on} & \type{YES/NO} & NO & 
If set "YES" program compute transport too.
\\ 
\hline
\key{Sorption} & \type{YES/NO} & NO & 
If set "YES" program include sorption too.
\\
\hline
\key{Dual\_porosity} & \type{YES/NO} & NO & 
If set "YES" program include dual porosity too.
\\
\hline
\key{Reactions} & \type{YES/NO} & NO & 
If set "YES" program include reactions too.
\\
\hline
\key{Concentration} & \type{string} & NULL &
Name of file with initial condition for concentrations of individual substances.
(see Section \ref{element_data_file})
\\ 
\hline
\key{Transport\_BCD} & \type{string} & NULL &
Name of file with boundary condition for transport.
(see Section \ref{transport_boundary_file})
\\ 
\hline
\key{Sources} & \type{string} & {\it undefined} &
Name of file with sources of species for transport.
(see Section \ref{element_data_file})
\\
\hline
\key{Transport\_out} & \type{string} & NULL &
Name of transport output file.
\\
\hline
\key{Transport\_out\_im} & \type{string} & NULL &
(NOT IMPLEMENTED IN 1.6.5) Name of transport immobile output file.
\\ 
\hline
\key{Transport\_out\_sorp} & \type{string} & NULL &
(NOT IMPLEMENTED IN 1.6.5) Name of transport sorbed output file.
\\ 
\hline
\key{Transport\_out\_im\_sorp} & \type{string} & NULL &
(NOT IMPLEMENTED IN 1.6.5) Name of transport sorbed immobile output file.
\\ 
\hline
\key{N\_substances} & \type{int} & -1 &
Number of substances.
\\
\hline
\key{Substances} & \type{string} & {\it undefined} &
\label{KeySubstanceNames}
Names of the substances separated by commas.
\\ 
\hline
%\key{Substances\_density\_scales} & \type{list of doubles} & 1.0 &
%Scales of substances for the density flow calculation.\\
%  \hline
\\ 
\key{bc\_times} & \type{list of doubles} & NULL &
Times for changing boundary conditions. If you set this variable, you have to prepare a separate file with boundary condintions for every 
time in the list. Filenames for individual time level are formed from BC filename by appending underscore and three digits of time level number, e.g. 
{\tt transport\_bcd\_000, transport\_bcd\_001, etc.} \\
\hline

\end{initable}
 
%\begin{initable}{Constants}
%\key{g} & \type{double} & 1.0 &
%Gravity acceleration.
%\\ 
%\hline
%\key{rho} & \type{double} & 1.0 &
%Density of fluid.
%\\
%\hline
%\end{initable}
 
% \normalsize
 

 
\begin{initable}{Solver}
\key{Use\_last\_solution} & \type{YES/NO} & NO &
If set to "YES", uses last known solution for chosen solver.
\\
\hline
\\
\key{Solver\_name} & \type{string} & petsc &
Type of linear solver.\br
Supported solvers are: {\tt petsc}, {\tt petsc\_matis} (experimental)
\\
\hline
\key{Solver\_params} & \type{string} & NULL & 
%Optional parameters for the external solver passed on the command line or
%PETSc options if the PETSc solver is chosen (see doc/petsc\_help). 
PETSc options to override default choice of iterative solver and preconditioner (use with care).
In particular to use UMFPACK sequantial direct solver set:

{\tt Solve\_params = "-ksp preonly -pc\_type lu -pc\_factor\_mat\_solver\_package umfpack" }

To use parallel direct solver MUMPS use:

{\tt Solve\_params = "-ksp preonly -pc\_type lu -pc\_factor\_mat\_solver\_package mumps -mat\_mumps\_icntl\_14 5"}
\\
\hline
\key{Keep\_solver\_files} & \type{YES/NO} & NO &
(NOT SUPPROTED IN 1.6.5) If set to "YES", files for solver are not deleted after the run of the solver.
\\
\hline
\key{Manual\_solver\_run} & \type{YES/NO} & NO &
(NOT SUPPROTED IN 1.6.5) If set to "YES", programm stops after writing input files for solver and lets user to run it.
\\ 
\hline
\key{Use\_control\_file} & \type{YES/NO} & NO &
(NOT SUPPROTED IN 1.6.5) If set to "YES", programm do not create control file for solver, it uses given file.
\\
\hline
\key{Control\_file} & \type{string} & NULL &
(NOT SUPPROTED IN 1.6.5) Name of control file for situation, when {\tt Use\_control\_file} \= YES.
\\
\hline
\key{NSchurs} & \type{int} & 2 &
Number of Schur complements to use. Valid values are 0,1,2. The last one should be the fastest.
\\
\hline
\key{Solver\_accuracy} & \type{double} & 1e-6 &
When to stop solver run - value of residum of matrix. 
Useful values from 1e-4 to 1e-10.\br
Bigger number = faster run, less accuracy.
\\
\hline
\key{max\_it} & \type{int} & 200 &
Maximum number of iteration of linear solver.
\\
\hline
\end{initable}
 
%\begin{initable}{Solver parameters}
%%
%%
%%  particular parameters for ISOL - reduce them
%%
%% method & string & fgmres & (i)\\
%%  \hline\\
%% stop\_crit & string & backerr &(i)\\
%%  \hline\\
%% be\_tol & double & 1e-10 &(i)\\
%%  \hline\\
%% stop\_check & int & 1 &(i)\\
%%  \hline\\
%% scaling & string & mc29\_30 &(i)\\
%%  \hline\\
%% precond & string & ilu &(i)\\
%%  \hline\\
%% sor\_omega & double & 1.0 &(i)\\
%%  \hline\\
%% ilu\_cpiv & int & 0 &(i)\\
%%  \hline\\
%% ilu\_droptol & double & 1e-3 &(i)\\
%%  \hline\\
%% ilu\_dskip & int & -1 &(i)\\
%%  \hline\\
%% ilu\_lfil & int & -1 &(i)\\
%%  \hline\\
%% ilu\_milu & int & 0 &(i)\\
%%  \hline\\
%\end{initable}
% Note: For aditional documentation see manual of the solver, (i) - isol manual
%%%%%%%%%%%%%%%%%%%%%%%%%%%%%%%%%%%%%%%%%%%%%%%%%%%%%%%%%%%%%%%%%%%%%%%%%%%%%%%%%%%%%%%%%%5 

\begin{initable}{Output}
\key{Write\_output\_file} & \type{YES/NO} & NO &
If set to "YES", writes output file.
\\
\hline
\key{output\_piezo\_head} & \type{YES/NO} & NO &
\label{KeyPiezoHead} 
If set to ``YES'', output also piezometric head into flow output files.
\\
\hline
\key{Output\_file} & \type{string} & NULL &
Name of the output file for water flow output.
\\
\hline
\key{Output\_file\_2} & \type{string} & NULL &
(NOT IMPLEMENTED IN 1.6.5) Name of the output file (type 2).
\\
\hline
\key{Output\_digits} & \type{int} & 6 &
(NOT IMPLEMENTED IN 1.6.5) Number of digits used for floating point numbers in output file.
\\
\hline
\key{Output\_file\_type} & \type{int} & 1 &
(NOT IMPLEMENTED IN 1.6.5)
Type of output file\br
 1 - GMSH like format\br
 2 - Flow data file\br
 3 - both files (two separate names)
\\
\hline
%\key{Write\_ftrans\_out} & \type{YES/NO} & NO &
%If set to "YES", writes output file for ftrans.
%\\
%\hline
%\key{Cross\_section} & \type{YES/NO} & NO &
%If set to "YES", uses cross section output.
%\\
%\hline
%\key{Cs\_params} & \type{double[7]} & {\it zero} &
%Params for cross section,\br
%[x0 y0 z0] initial point\br
%[xe ye ze] end point\br
%[delta] cylinder radius.
%\\
%\hline\\
%\key{Specify\_elm\_type} & \type{YES/NO} & NO &
%If set to "YES", next param. specify type of prefered elements. If set to
%"NO", each element is included.
%\\ 
%\hline
%\key{Output\_elm\_type} & \type{int} & -1 &
%Spefify type of element dimension\br
%1 - 1D (line), 2 - 2D (triangle), \br
%3 - 3D (tetrahedron).
%\\
%\hline
%\key{BTC\_elms} & \type{list of ints} & {\it undefined} &
%List of the breakthrough curve elements, ints this concentrations are written to
%seperate file with extension *.btc.
%\\
%\hline\\
%\key{FCs\_params}& double[4] & {\it zero} &
%Params of flow cross section\br
%[x y z 1] plane of cut (general equation),\br
% output values are written by coordinate\br
% of axis: x - [0], y - [1], z - [2]
%\\
%\hline
\key{Pos\_format} & \type{string} & ASCII &
\label{KeyOutFormat}
Output file format. One can use: ASCII, BIN, or VTK\_SERIAL\_ASCII
\\
\hline\\
\key{balance\_output} & \type{string} & NULL &
Name of file for output of water boundary fluxes and balnace of sources over material subdomains.
\\
\hline
\end{initable}

% \begin{initable}{Density}
% \key{Density\_implicit} & \type{YES/NO} & NO &
% NO = explicit iteration (simple flow update)\br
% YES = implicit iteration (more accurate flow update)
%\\
%\hline
%\key{Density\_max\_iter} & int & 20 &
%Maximum number of iterations for implicit density calcultation.
%\\
%\hline\\
%\key{Eps\_iter} & \type{double} & 1e-5&
%Stopping criterium for iterations (maximum norm of pressure difference).
%\\
%\hline
%\key{Write\_iterations} & \type{YES/NO} & NO &
%Write conc values during iterations to POS file.
%\\
%\hline\\
%\end{initable}

\begin{initable}{Semchem\_module}
\key{Compute\_reactions} & \type{Yes/No} & "No" &
NO = transport without chemical reactions\br
YES = transport influenced by chemical reactions
\\
\hline
\key{Output\_precission} & \type{int} & 1 &
Number of decimal places written to output file created by Semchem\_module.
\\
\hline
\key{Number\_of\_further\_species} & \type{int} & 0 &
Concentrations of these species are not computed, because they are ment to be unexghaustible.
\\
\hline
\key{Temperature} & \type{double} & 0.0 &
Temperature, one of state variables of the system.
\\
\hline
\key{Temperature\_Gf} & \type{double} & 0.0 &
Temperature at which Free Gibbs Energy is specified.
\\
\hline
\key{Param\_Afi} & \type{double} & 0.0 &
Parameter of the Debuy-H\"{u}ckel equation for activity coeficients computation.
\\
\hline
\key{Param\_b} & \type{double} & 0.0 &
Parameter of the Debuy-H\"{u}ckel equation for activity coeficients computation.
\\
\hline
\key{Epsilon} & \type{double} & 0.0 &
Epsilon specifies relative norm of residuum estimate to stop numerical algorithms used by Semchem\_module.
\\
\hline
\key{Time\_steps} & \type{int} & 1 &
Number of transport step subdivisions for Semchem\_module.
\\
\hline
\key{Slow\_kinetics\_substeps} & \type{int} & 0 &
Number of substeps performed by Runge-Kutta method used for slow kinetics simulation.
\\
\hline\\
\key{Error\_norm\_type} & \type{string} & "Absolute" &
Through wich kind of norm the error is measured.
\\
\hline\\
\key{Scalling} & \type{boolean} & "No" &
Type of the problem preconditioning for better convergence of numerical method.
\\
\hline\\
\end{initable}
\newpage
\begin{initable}{Aqueous\_species}
\key{El\_charge} & \type{int} & 0 &
Electric charge of an Aqueous\_specie particleunder consideration.
\\
\hline
\key{dGf} & \type{double} & 0.0 &
Free Gibbs Energy valid for TemperatureGf.
\\
\hline
\key{dHf} & \type{double} & 0.0 &
Enthalpy
\\
\hline
\key{Molar\_mass} & \type{double} & 0.0 &
Molar mass of Aqueous\_species.
\\
\hline
\end{initable}

\begin{initable}{Further\_species}
\key{Specie\_name} & \type{string} & "" &
Name belonging to Further\_specie under consideration.
\\
\hline
\key{dGf} & \type{double} & 0.0 &
Free Gibbs Energy valid for TemperatureGf.
\\
\hline
\key{dHf} & \type{double} & 0.0 &
Enthalpy
\\
\hline
\key{Molar\_mass} & \type{double} & 0.0 &
Molar mass of Further\_species.
\\
\hline
\key{Activity} & \type{double} & 0.0 &
Activity of Further\_species.
\\
\hline
\end{initable}

\begin{initable}{Reaction\_i}
\key{Reaction\_type} & \type{string} & "unknown" &
Type of considered reaction (Equilibrium, Kinetics, Slow\_kinetics).
\\
\hline
\key{Stoichiometry} & \type{int} & 0 &
Stoichiometric coeficients of species taking part in $i$-th reaction.
\\
\hline
\key{Kinetic\_constant} & \type{double} & 0.0 &
Kinetic constant for determination of reaction rate.
\\
\hline
\key{Order\_of\_reaction} & \type{int} & 0 &
Order of kinetic reaction for participating species.
\\
\hline
\key{Equilibrium\_constant} & \type{double} & 0.0 &
Equilibrium constant defining i-th reaction.
\\
\hline
\end{initable}

\begin{initable}{Reaction\_module}
\key{Compute\_decay} & \type{Yes/No} & "No" &
It enables to switch on simulation of radioactive decay or first order reactions.
\\
\hline
\key{Nr\_of\_decay\_chains} & \type{int} & 0 &
How many decay chains are considered.
\\
\hline
\key{Nr\_of\_FoR} & \type{int} & 0 &
How many first order reactions are defined.
\\
\hline
\end{initable}

\begin{initable}{Decay\_i}
\key{Nr\_of\_isotopes} & \type{int} & 0 &
It defines the number of isotopes which does the current section [Decay\_i] work with.
\\
\hline
\key{Substance\_ids} & \type{array of int} & NULL &
Sequence of ids describing the order of isotopes in decay chain.
\\
\hline
\key{Half\_lives} & \type{array of double} & NULL &
Contain half-lives belonging to isotopes defined by ids.
\\
\hline
\key{Bifurcation\_on} & \type{Yes/No} & "No" &
It makes it possible to define branches in current [Decay\_i].
\\
\hline
\key{Bifurcation} & \type{array of double} & NULL &
It defines a percentage, which is the first isotope in current [Decay\_i] decaying to products.
\\
\hline
\end{initable}

\begin{initable}{FoReact\_i}
\key{Kinetic\_constant} & \type{double} & 0.0 &
It defines kinetic constant which is the corresponding half-life computed from.
\\
\hline
\key{Substance\_ids} & \type{array of int} & NULL &
It containes a couple of indices of substances which are taking part in first order reactions described in current section FoReact\_i.
\\
\hline
\end{initable}

\begin{initable}{Checkpointing\_i}
\key{Checkpointing\_on} & \type{Yes/No} & "No" &
Checkpointing is enabled / disabled\br
YES - Checkpointing is enabled\br
NO - Checkpointing is disabled.
\\
\hline
\key{Checkpoints_type} & \type{array of int} & FIXED &
How are Checkpoints generated:
FIXED - Checkpoints are in fixed times, based on Stop_time, Save_step and Number_of_checkpoints.
DYNAMIC - Checkpoints are generated in real time intervals, based on Checkpoint_interval.
\\
\hline
\key{Number_of_checkpoints} & \type{int} & 0 &
How many Checkpoints is generated when FIXED Checkpoints is selected.\br 
Maximum stored checkpoints is 10.
\\
\hline
\key{Checkpoints_interval} & \type{int} & 0 &
Time interval for Checkpoints.\br 
How often is checkpoint stored.[seconds]
\\
\hline
\key{Max_checkpoints} & \type{int} & 0 &
How many Checkpoints is stored.\br 
Maximum stored checkpoints is 10.
\\
\hline
\key{Output_file_format} & \type{string} & TXT &
Type of output file format.\br
TXT - output file format is txt.\br
BIN - output file format is binary.
\\
\hline
\key{Output_file_prefix} & \type{string} & chp_ &
File prefix for output files.
\\
\hline
\key{Output_file_suffix} & \type{string} & .dat &
File suffix for output files.
\\
\hline
\end{initable}

\vfill
\pagebreak