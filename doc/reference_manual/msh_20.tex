% Copyright (C) 2007 Technical University of Liberec.  All rights reserved.
%
% Please make a following refer to Flow123d on your project site if you use the program for any purpose,
% especially for academic research:
% Flow123d, Research Centre: Advanced Remedial Technologies, Technical University of Liberec, Czech Republic
%
% This program is free software; you can redistribute it and/or modify it under the terms
% of the GNU General Public License version 3 as published by the Free Software Foundation.
%
% This program is distributed in the hope that it will be useful, but WITHOUT ANY WARRANTY;
% without even the implied warranty of MERCHANTABILITY or FITNESS FOR A PARTICULAR PURPOSE.
% See the GNU General Public License for more details.
%
% You should have received a copy of the GNU General Public License along with this program; if not,
% write to the Free Software Foundation, Inc., 59 Temple Place - Suite 330, Boston, MA 021110-1307, USA.


\subsection{Mesh file format version 2.0}
The only supported format for the computational mesh is MSH ASCII format produced 
by the GMSH software. You can find its documentation on:

\url{http://geuz.org/gmsh/doc/texinfo/gmsh.html#MSH-ASCII-file-format}


Comments concerning Flow123d:
\begin{itemize}
  \item Every inconsistency of the file stops the calculation.
    These are:
      \begin{itemize}
        \item Existence of nodes with the same \vari{node-number}.
        \item Existence of elements with the same \vari{elm-number}.
        \item Reference to non-existing node.
        \item Reference to non-existing material (see below).
        \item Difference between \vari{number-of-nodes} and actual number of
          lines in nodes' section.
        \item Difference between \vari{number-of-elements} and actual number of
          lines in elements' section.
      \end{itemize}
  \item By default Flow123d assumes meshes with \vari{number-of-tags} = 3.
    \begin{description}
    \item[\vari{tag1}] is number of geometry region in which the element lies. 
    \item[\vari{tag2}] is number of material (reference to {\tt
    .MTR} file) in the element.
    \item[\vari{tag3}] is partition number (CURRENTLY NOT USED).
    \end{description}
  \item Currently, line (\vari{type} = 1), triangle (\vari{type} = 2) and
    tetrahedron (\vari{type} = 4) are the only supported types
    of elements. Existence of an element of different type stops the calculation.
  \item Wherever possible, we use the file extension {\tt .MSH}. It is not
    required, but highly recomended.
\end{itemize}

%%%%%%%%%%%%%%%%%%%%%%%%%%%%%%%%%%%%%%%%%%%%%%%%%%%%%%%%%%%%%%%%%%%%%%%%%%%%%%
% vim: set tw=78 ts=2 sw=2 expandtab nocindent smartindent:
%%%%%%%%%%%%%%%%%%%%%%%%%%%%%%%%%%%%%%%%%%%%%%%%%%%%%%%%%%%%%%%%%%%%%%%%%%%%%%


